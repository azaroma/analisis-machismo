%
% API Documentation for Project
% Module analisis-machismo.app.analysis
%
% Generated by epydoc 3.0.1
% [Fri Dec  9 04:12:59 2016]
%

%%%%%%%%%%%%%%%%%%%%%%%%%%%%%%%%%%%%%%%%%%%%%%%%%%%%%%%%%%%%%%%%%%%%%%%%%%%
%%                          Module Description                           %%
%%%%%%%%%%%%%%%%%%%%%%%%%%%%%%%%%%%%%%%%%%%%%%%%%%%%%%%%%%%%%%%%%%%%%%%%%%%

    \index{analisis-machismo \textit{(package)}!analisis-machismo.app \textit{(package)}!analisis-machismo.app.analysis \textit{(module)}|(}
\section{Module analisis-machismo.app.analysis}

    \label{analisis-machismo:app:analysis}

%%%%%%%%%%%%%%%%%%%%%%%%%%%%%%%%%%%%%%%%%%%%%%%%%%%%%%%%%%%%%%%%%%%%%%%%%%%
%%                               Functions                               %%
%%%%%%%%%%%%%%%%%%%%%%%%%%%%%%%%%%%%%%%%%%%%%%%%%%%%%%%%%%%%%%%%%%%%%%%%%%%

  \subsection{Functions}

    \label{analisis-machismo:app:analysis:main}
    \index{analisis-machismo \textit{(package)}!analisis-machismo.app \textit{(package)}!analisis-machismo.app.analysis \textit{(module)}!analisis-machismo.app.analysis.main \textit{(function)}}

    \vspace{0.5ex}

\hspace{.8\funcindent}\begin{boxedminipage}{\funcwidth}

    \raggedright \textbf{main}(\textit{keys}={\tt 'keys.ini'}, \textit{raw\_tweets\_file}={\tt 'twitter\_data.txt'}, \textit{no\_tweets}={\tt 1000}, \textit{tracked\_words\_file}={\tt 'tracks.csv'}, \textit{formatted\_tweets\_file}={\tt 'formatted\_tweets.txt'}, \textit{dictionaries}={\tt ['misoginy\_dictionary.yml','curses\_dictionary.yml']})

    \vspace{-1.5ex}

    \rule{\textwidth}{0.5\fboxrule}
\setlength{\parskip}{2ex}
    Perform an analyisis to find sexist and rude words in tweets

    This module employs every other module to perform a full analysis on 
    data retrieved from the Twitter stream. First a TwitterMiner retrieves 
    data and dumps it, then a TweetFormatter parses the data into a list of
    tweets that are lists of words. Then it uses the spaghetti tagger to 
    POStag every word, yielding a list of tweets that are lists with 
    elements with the form (word, [tags]). A DictionaryTagger adds our 
    custom tags to the [tags] list. Finally a TagCounter perform a count of
    every tag found in tweets. This program prints the number of 
    coincidences of our custom tags.

\setlength{\parskip}{1ex}
    \end{boxedminipage}

    \index{analisis-machismo \textit{(package)}!analisis-machismo.app \textit{(package)}!analisis-machismo.app.analysis \textit{(module)}|)}
