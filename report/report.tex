\documentclass{scrartcl}
\usepackage{polyglossia}
\setmainlanguage{spanish}

\title {Detección de sexismo en Twitter \\ Ingeniería de Software}
\author{Abraham Solís Álvarez \\ Mario Alejandro Gil Lázaro}

\begin{document} \maketitle \tableofcontents

\section{Descripción}
Este proyecto busca determinar los niveles de sexismo y otras manifestaciones
de odio y discriminación, en el texto que los usuarios del sitio de
microblogeo Twitter, escriben diariamente. Dado que el internet es una
plataforma moderna de expresión y debido también a que la ya mencionada red
social posee un número importante de usuarios, consideramos que la información
ahí recabada representa una muestra importante. Así mismo, el hecho de que
los medios virtuales son comúnmente considerados como medios poco
trascendentales, en los que se puede supuestamente evitar las consecuencias
que los propios comentarios puedan ocasionar, creemos que las opiniones ahí
expresadas poseen un alto grado de sinceridad, mayor al que podría obtenerse
en el discurso hablado.

\section{Requerimientos}
El sistema propuesto debe ser capaz de recolectar los posts directamente del
stream de Twitter, convertir los datos al formato que mejor convenga, etiquetar
palabra por palabra el texto, y realizar mediciones que permitan elaborar
indicadores que arrojen luz sobre las costumbres y la cultura de los usuarios
de la red social. Se pretende lograr esto manteniendo los más altos estándares
de calidad que sea posible, aplicando prácticas de programación modernas.

\section{Dependencias}
Para ejecutar el programa con las mayores probabilidades de éxito se recomienda
utilizar un sistema operativo basado en GNU/Linux. Es necesario instalar una
Python 3.x, de preferencia la versión 3.5, que fue la utilizada en el
desarrollo del sistema. Se recomienda utilizar el programa `pip` para instalar
los módulos de Python `nltk` y `plac`, que son necesarios para ejecutar el
programa. Es necesario contar con una conexión a internet.

\section{Modo de uso}
Por el momento, la única forma de interactuar con el sistema es mediante el
módulo `analysis.py`, localizado en el directorio `app` (con el comando
`python analysis.py`). Tenga en cuenta que usted debe estar localizado dentro
de dicho directorio. Además, se proporciona un script que ejecuta las pruebas
del sistema en el directorio raíz: `run_tests`.

\end{document}
%%% Local Variables:
%%% mode: latex
%%% TeX-master: t
%%% End:
